\documentclass{article}
\usepackage{amsmath, amssymb}
\title{Certificate of Reflection Positivity for Wilson Lattice Gauge Theory}
\author{Yang-Mills Verification Agent}
\date{\today}
\begin{document}
\maketitle

\section{Theorem Statement}
The Wilson Plaquette Action $S_W$ on a hypercubic lattice $\Lambda \subset \mathbb{Z}^4$ with gauge group $G=SU(3)$ satisfies Reflection Positivity with respect to lattice hyperplane reflections.

\section{Formal Definition}
Let $\theta$ be the reflection operator across a time-slice $x_0 = 0$.
The expectation value is defined by:
$$ \langle F \rangle = \frac{1}{Z} \int \mathcal{D}U \, F[U] e^{-S_W[U]} $$
For any observable $A$ supported on the positive time half-lattice $\Lambda_+$, we define $\Theta A$ as the antilinear reflection of $A$.
The Reflection Positivity condition requires:
$$ \langle (\Theta A) A \rangle \ge 0 $$
for all such $A$.

\section{Proof Reference}
This result is standard for the Wilson action.
Reference: Osterwalder, K., & Seiler, E. (1978). "Gauge Field Theories on the Lattice". Annals of Physics.
The measure is a product of Haar measures, which is symmetric. The interaction term involves traces of variables around plaquettes. Plaquettes not crossing the reflection plane are mapped to themselves (or their partners). Plaquettes crossing the plane can be written in the form $Tr(L U R^\dagger)$.
The exponential of the action can be expanded in characters, ensuring positivity.

\section{Verification Status}
This file serves as the formal artifact referenced by `rp_evidence.json`.
\end{document}
